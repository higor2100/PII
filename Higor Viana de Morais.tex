
\documentclass[14pt, a2paper, portrait]{tikzposter}

\usepackage[brazil]{babel}
\usepackage[utf8]{inputenc}
\usepackage[T1]{fontenc}
\usepackage{lipsum}

 
\newcommand{\bs}{\textbackslash}   
\newcommand{\cmd}[1]{{\bf \color{red}#1}}   
\tikzposterlatexaffectionproofon

\usepackage{xcolor}
 \definecolor{Guardian}{RGB}{43, 165, 219}

\title{Pôster no {\LaTeX}}
\author{Henrique Ferreira Rocumback\\
	\\Henrique Monteiro Martins
	\\Higor Viana de Morais
	\\Vinicius Carvalho Venancio da Silva
	\\Ygor Kaique de Souza Pinto\\ 
\texttt{higor.2100@outlook.com}}

\institute{Centro Universitário Senac}

 % Set colortheme
 % (default, anil, armin, edgar, emre, hanna, james, kai, lena, manuel,
 % martin, max, nicolas, pascal, peter, philipp, richard, roman, stefanie,
 % vinay)
\usecolortheme{lena}

\definecolor{framecolor}{named}{black}

\settitlebodystyle{rectangular}
\setblocktitlestyle{rounded}
\setblockbodystyle{shaded}

\begin{document}
\titleblock[left fig=logo.png, embedded]
\block[l]{Introdu\c{c}\~ao}{
A franquia de jogos da grande Nintendo chamada The Legend of Zelda, vem trazendo f\~as por gera\c{c}\~oes e como um f\~a de The Legend of Zelda, decidi fazer uma homenagem a essa que \'e uma das melhores franquias. The Legend of Zelda nos remete as nossas inf\^ancias com seus jogos e \'e essa a proposta de construir um Guardian de The Legend of Zelda - Breath of the Wild, nos lembrar que um jogo pode ficar guardado em nossas memorias e em nossos cora\c{c}\~oes.
}

\begin{columns}
% 1a coluna
\column{0.48}

\block[c]{Materiais e Métodos}{
\block[c]{Resultado}{
  \begin{tikzpicture}
  \tikzbwpixelart[color=Guardian, scale=.2]{(0,0)}{
   0000000000000100000010000000000000
0000000000010110000110100000000000
0000000000111111111111110000000000
0000000000111111111111110000000000
0000000000011111111111100000000000
0000000000001111111111000000000000
0000000000001111111111000000000000
0000000000001111111111000000000000
0000000000001111111111000000000000
0000000000011111111111100000000000
0000000000011111111111100000000000
0000000000011111111111100000000000
0000000000011111111111100000000000
0000000000011111111111100000000000
0000000000000000000000000000000000
0000000000011111111111100000000000
0000000000000000000000000000000000
0000000010000111111110001110000000
0000001111000011111100011111000000
0000001111100001111000110111000000
0000011111100000110010111101100000
0000011101101010110100000110110000
0000100011101010110100111010110000
0001111111010010000100100011111000
0000011100110010000110100010000000
0000000011110100000010011110000000
0001111011100101001011000000110100
0001110100001011001101000011011010
0101110011110010000110101100001010
0101011011110101001010101101001100
1100001011110100000010001101001100
1100001010010000000000001101001111
1100010110000000001011000110110111
0110000110011100110011111000001110
0111111111111110000111111111111110
0011111111111111111111111111111100
0000011111111111111111111111100000
0000000000000000000000000000010000
0000011111111111111111111111100000
0000001111111111111111111111000000
0000000111111111111111111110000000
 }
 %
    \draw (4, 1) node[opacity=0]{\LARGE PixelArt};
\end{tikzpicture}
}
\begin{center}
  \begin{minipage}{\linewidth}
    \centering
    \includegraphics[scale=1]{1.png}
  \end{minipage}
\end{center}
}
% 2a coluna
\column{0.52}

\block[c]{Resultado}{
  \begin{tikzpicture}
  \tikzbwpixelart[color=Guardian, scale=.2]{(0,0)}{
   0000000000000100000010000000000000
0000000000010110000110100000000000
0000000000111111111111110000000000
0000000000111111111111110000000000
0000000000011111111111100000000000
0000000000001111111111000000000000
0000000000001111111111000000000000
0000000000001111111111000000000000
0000000000001111111111000000000000
0000000000011111111111100000000000
0000000000011111111111100000000000
0000000000011111111111100000000000
0000000000011111111111100000000000
0000000000011111111111100000000000
0000000000000000000000000000000000
0000000000011111111111100000000000
0000000000000000000000000000000000
0000000010000111111110001110000000
0000001111000011111100011111000000
0000001111100001111000110111000000
0000011111100000110010111101100000
0000011101101010110100000110110000
0000100011101010110100111010110000
0001111111010010000100100011111000
0000011100110010000110100010000000
0000000011110100000010011110000000
0001111011100101001011000000110100
0001110100001011001101000011011010
0101110011110010000110101100001010
0101011011110101001010101101001100
1100001011110100000010001101001100
1100001010010000000000001101001111
1100010110000000001011000110110111
0110000110011100110011111000001110
0111111111111110000111111111111110
0011111111111111111111111111111100
0000011111111111111111111111100000
0000000000000000000000000000010000
0000011111111111111111111111100000
0000001111111111111111111111000000
0000000111111111111111111110000000
 }
 %
    \draw (4, 1) node[opacity=0]{\LARGE PixelArt};
\end{tikzpicture}
}

\block{Considerações Finais}{
Foi muito interessante fazer a cria\c{c}\~ao no pacote PixelArt e da base a muito conhecimento como parte de um contexto geral do Conte\'udo do Semestre.
}
\end{columns}

\block[c,width=30cm]{Referências}{
\begingroup
   \renewcommand{\section}[2]{}
   \begin{thebibliography}{10}
	
	    \bibitem{Oetiker} OETIKER, Tobias et. al. {\sl Introdução ao {\LaTeXe}}, 2001.

	    \bibitem{Tantau} TANTAU, Till. {\sl The TikZ and PGF Packages}, http://www.texample.net/tikz/, 2014.
	    
	    \bibitem{PixelArt} Louis Pa­ter­nault {\sl The Graphics}, https://ctan.org/tex-archive/graphics/pixelart, 2018.
	    
	    \bibitem{Nintendo} The Legend of Zelda - Breath of the Wild™  {\sl The Game},https://www.zelda.com/breath-of-the-wild/, 2017.
			
			\bibitem{Richter} RICHTER, Pascal et. al. {\sl The TikZposter class}, http://www.ctan.org/pkg/tikzposter/, 2014.

   \end{thebibliography}
\endgroup
}

\end{document}


